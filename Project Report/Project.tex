% -------------------------------------------------------------
% CONFIGURE FILE
\documentclass{report}
\setlength{\parindent}{0pt}
% -------------------------------------------------------------
% PACKAGES 
\usepackage{xcolor}
\usepackage[normalem]{ulem} 
\usepackage{graphicx}
\usepackage{cite}
\usepackage{amsmath}
\usepackage{amssymb}
\usepackage{array}
\usepackage{parskip}
\usepackage{booktabs}
\usepackage[hidelinks]{hyperref}
\usepackage[justification=centering]{caption}
% -------------------------------------------------------------
% START DOCUMENT CONTENT
% -------------------------------------------------------------
\begin{document}   
\pagecolor{white}
% -------------------------------------------------------------
% COVER PAGE
% -------------------------------------------------------------
\begin{center}
	\includegraphics[scale=0.5]{lu-logo}

	{\vspace{3em}}

	\Huge{{\textbf{Sentiment Analysis of Twitter to Optimise Fantasy Premier League Performance}}}

	{\vspace{1em}}

	\LARGE{By Lewis Watt}

	{\vspace{2em}

	\Large{
	Loughborough University\\
	Student ID: F125967\\
	17COC251: Computer Science Project}

	{\vspace{2em}}

	\Large{
	Supervisor: Magda Zajaczkowska\\
	Submitted: }}
\end{center}
% -------------------------------------------------------------
% CONTENTS
% -------------------------------------------------------------                          
\tableofcontents 
% -------------------------------------------------------------
% INTRODUCTION
% -------------------------------------------------------------
\chapter[Introduction]{Introduction}
% -------------------------------------------------------------
% BACKGROUND
% -------------------------------------------------------------
\section[Background]{Background}
Fantasy Premier League (FPL) is a popular online sports game based around the real-life top tier of English Football, the Premier League. Players of the game, often referred to as 'managers', are tasked with selecting a squad of 15 real-life premier league players every week. In game players are assigned a monetary value ranging from £4-15m, and each manager has a budget of £100m to build their team. The real-life performance of players determines how many points they are rewarded each week. Positive actions like scoring goals and providing assists increase points, whilst negative actions like receiving a red card or scoring an own-goal result in a decrease in points. Over the course of a 38-gameweek season, managers aim to score the most amount of points by rotating players in and out of their teams as they see fit.

The first version of FPL was launched in 2002 alongside the launch of the Premier League website, ahead of the 2002/3 season. Around 75,000 players participated in the first season \cite{fpl-origins}, and since then growth has been explosive, with the player count surpassing 10 million in the 2022/23 season \cite{fpl-playercount}. Each season the game has become more competitive, with the Premier League offering incentives such as a 7-night break inclusive of VIP hospitality tickets at two 2025/26 Premier League matches for the winner of the current season \cite{fpl-prizes}. As well as the official prizes, many people participate in 'money leagues' where a small fee is paid for entry, and the winner takes home the prize pot. Alongside this more casual betting, professional gambling companies have started offering fantasy themed games where participants often stake money on day or weekend long periods, choosing a fantasy team to try and win them money by scoring the most points. With the global fantasy sports market valued at over \$27 billion in 2022, and projected to reach \$87 billion in 2031\cite{fpl-market-growth}, it is no surprise companies are scrambling to try and capitalise off of the immense popularity of the game.

With the increased popularity of FPL and a rapidly growing market full of potential customers, AI-based platforms aimed at increasing customers' overall ranks for little effort have started to emerge. These tools offer managers a quick and easy way to put together a team that they know will perform at a decent level, taking the effort out of the game for the managers who don't want to spend time diving through data and stats. AI tools also take the bias out of the game, as people tend to just pick their favourite players or players from their favourite teams, regardless of how well they are likely to perform in the game. Whether it be for monetary reasons, or for the simple pleasure of getting bragging rights over friends, the use of these platforms has shot up particularly in the last 2 years. One of the biggest AI platforms is \textit{Fantasy Football Hub} \cite{ffh}, which launched in 2019 and has grown to over 40,000 paying users, with around £2.5m in annual revenue \cite{ffh-users}. The platform uses a longitudinal multilevel regression model \cite{multilevel-models} in combination with the football data platform \textit{Opta} to analyse each player's potential. The model is used to offer users of the platform recommendations for transfers, alongside a spreadsheet of points players are expected to score each week. By using this information to deliver customers clear and precise recommendations, \textit{Fantasy Football Hub} has been successful in capturing a large customer base who can market their platform through word of mouth.

Another tool people consult when making FPL related decisions is social media, where users often share their teams with each other and discuss potential players to buy or sell. It was found that over 70\% of FPL players find social media to be at least somewhat influential when it comes to making decisions around their team \cite{whittaker}. The FPL subreddit r/FantasyPL has over 700k members \cite{r/FPL}, and the official FPL Twitter account has a whopping 6.2m followers \cite{@OfficialFPL}. As well as the official accounts run by the Premier League themselves, many so called 'experts' have started to gain a large following on Twitter. Accounts such as \texttt{@FPLGeneral}, \texttt{@FFScout}, \texttt{@BenCrellin}, and \texttt{@LetsTalk\_FPL} all have well over 250k followers, where useful tips and insights are often posted to help followers decide who to buy and sell each week.

However, using social media can often be a tedious process, sifting through thousands of posts trying to decide which ones to listen to - a stark contrast to the ease of use AI platforms bring. By leveraging sentiment analysis, this manual process can be automated, enabling AI to sift through vast amounts of data and extract meaningful insights. This study aims to explore how sentiment analysis can be applied to social media data and used in combination with existing AI models to enhance the accuracy of recommendations for FPL managers. In doing so it adds to the ongoing discussion around integrating human feedback into existing AI models used to forecast FPL performance. This study hopes to explore new areas with a particular emphasis on Twitter, a platform previously neglected when it comes to FPL research, due to the complicated jargon found in typical posts.
% -------------------------------------------------------------
% FPL RULES
% -------------------------------------------------------------
\section[Rules of FPL]{Rules of FPL}
Note the terms \textit{player} and \textit{manager} are often confused when it comes to FPL, so please be aware that this study uses the term \textit{manager} to refer to people who participate in the game of FPL, and \textit{players} to refer to real-life premier league footballers.

Fantasy Premier League is a mirror of the real-life English Premier League, meaning that each in game event (referred to as a \textit{gameweek}) represents a set of real-life football matches. The game takes place over a \textit{season} which usually runs from August until May of the following calendar year, and at the end of the season the winner is crowned. There are 20 teams who play each other twice over the course of the season, resulting in a total of 38 gameweeks. Note that sometimes due to unforeseen circumstances like bad weather or domestic cup fixtures, games are postponed resulting in a team not playing in one gameweek (known as a \textit{blank} gameweek), and playing twice in another gameweek (known as a \textit{double} gameweek) to make up.

Upon the start of the season, each manager is tasked with choosing a team of players from a database of around 500. A team must be made up of 15 players, consisting of 2 goalkeepers (GKs), 5 defenders (DEFs), 5 midfielders (MIDs), and 3 strikers (STs). On top of these positional constraints, each player is assigned a monetary value in £m's (roughly according to how well they performed in the previous season), and the manager is given a budget of £100m from which they must select their team. A final constraint on team selection is that you cannot select more than 3 players from the same team, so it would not be possible to buy the whole Manchester United team, for example.

	% -------------------------------------------------------------
	%INITIAL TEAM SELECTION FIGURE
	% -------------------------------------------------------------
	\begin{figure}[htbp]
	\begin{center}
	\includegraphics[scale=0.275]{team-selection-initial}
	\caption{Initial Team Selection}
	\label{default}
	\end{center}
	\end{figure}
	% -------------------------------------------------------------
	%INITIAL TEAM SELECTION FIGURE
	% -------------------------------------------------------------
	
Replacing a player in your team with another is known as a \textit{transfer}. Each manager is given 1 free transfer (\textit{FT}) every gameweek, which can accumulate up to a maximum of 5. Managers who make a number of transfers that is higher than their free transfer balance must pay 4 points per additional transfer - referred to as a \textit{hit}. Transfers cannot lead to illegal teams, i.e. teams resulting from transfers must still meet the positional, budgetary, and 3-player per team constraints mentioned above.

Another variable affecting team selection is price changes. Throughout the season player's values will go up and down depending on how popular they are among the player base. The more people buying a player, the more expensive they become and vice versa. This means managers can grow their budget by selling players who have gone up in value. When a manager sells a player who has gone up in value, the manager receives 50\% of that increase rounded down to the nearest £0.1m. So if a manager bought a player for £5m and sells them at £5.5m, the manager will only receive £5.2m back. However, when a player goes down in value, the manager suffers the full price loss.

Every gameweek, managers must choose 11 \textit{starters} (a \textit{starting 11}) and 4 \textit{substitutes} or \textit{'subs'}. The starting 11 is the set of players who contribute to a manager's overall score (points scored by substitutes do not count towards their total). A starting 11 must consist of at least 3 DEFs, 3 MIDs, 1 ST, and exactly 1 GK. Subs will replace starters in the event that the starting player does not play a single minute in their real-life match. Before the start of the gameweek, the manager must choose a substitution preference, which is the order in which substitutes replace starters. This preference does not supersede the starting 11 constraints - that is, if a defender did not play in real-life and a midfielder was first in the substitute preference, the midfielder would not replace the defender if it resulted in a team with less than 3 defenders.

One of the most important choices a manger must make is who to \textit{captain}. Giving a player captaincy places a 2x multiplier on their points, meaning if they were to score 10 points the manager would receive 20. A vice-captain is also chosen, and in the case that the captain is replaced by a substitute they will become the new captain.

A gameweek \textit{deadline} is the final opportunity for managers to make changes to their teams for that gameweek. This deadline always occurs 90 minutes before the first real-life match of the gameweek. Once the deadline passes, any changes made will apply only to the following gameweek. Points are then awarded based on the outcomes of the matches over the next few days. This process of team selection, deadlines, and scoring continues throughout the season, resulting in a leaderboard ranking that determines the winner. The points scoring system can be found \href{https://www.premierleague.com/news/2174909}{\textcolor{blue}{\uline{here}}}
% -------------------------------------------------------------
% PROJECT AIMS AND OBJECTVES
% -------------------------------------------------------------
\section[Project Aims and Objectives]{Project Aims and Objectives}
\subsection[Goals]{Goals}
The main goal of the project is to create a system that will take twitter data about English Premier League players as an input, and use a fine-tuned sentiment analysis model to analyse the twitter data. The sentiment for each player should be used to create an optimised forecast of a player's expected points (xP) for each \textit{gameweek}. The system will then use a team selection algorithm to pick an optimal squad based on each player's xP and price. The team selection algorithm will also have to abide by the constraints placed on team selection by the official FPL rules.

Users should be able to interact with the system using commands via the console, however the data should be well organised to make it easy to integrate into a web or mobile app with a fully functioning user interface in the future. The system will be modular, meaning the sentiment analysis model, squad selection algorithm, and xP prediction models should be easily upgradable. This should allow the system to keep up with advancements in forecasting technologies that may arise in future.

To the knowledge of the author, no research has been done surrounding sentiment analysis of twitter for player performance forecasting in the FPL domain. This study aims to advance research in this area and further contribute to the existing fantasy sports forecasting research. The results of the system will be tested against existing research done with no sentiment analysis and the most popular AI FPL platform that charges users a subscription fee to use its services. As well as this, models that have used sentiment analysis of other forms of media such as news articles will be compared to test the suitability of using twitter as a data source. The system will be simulated over a range of gameweeks and its overall score will be compared to see where it ranks in the global leaderboard against other prediction methods.
\subsection[Objectives]{Objectives}
\begin{enumerate}
    \item Analyse the project aims and break them down into key functional and non-functional requirements.
    \item Conduct a literature review on fantasy sports forecasting and sentiment analysis techniques, as well as research on existing FPL forecasting platforms.
    \item Fine-tune a sentiment analysis model specifically for capturing player sentiment in the FPL context, using labeled data for model training and validation.
    \item Develop and validate a predictive model that uses Twitter sentiment, historical player data, and match fixtures to forecast a player's expected points (xP) for each gameweek.
    \item Implement a team selection algorithm that optimises squad selection based on player xP, prices, and FPL constraints, ensuring compatibility with official game rules.
    \item Test and evaluate the system by simulating its performance over multiple gameweeks against existing FPL forecasting methods, both with and without sentiment analysis.
    \item Prepare a comprehensive report documenting the findings, challenges, and recommendations for future work by the end of the project.
\end{enumerate}

\subsection[Statistical Objectives]{Statistical Objectives}
\begin{itemize}
    \item The main goal of the system is to achieve a higher overall rank than existing models over the course of an FPL season.
    \item Evaluate the effectiveness of Twitter sentiment analysis in improving xP prediction accuracy by comparing the mean squared error (MSE) or root mean squared error (RMSE) of models with and without sentiment analysis.
    \item Assess the impact of the sentiment analysis model on team performance by comparing the total points scored by squads selected with and without sentiment-informed xP predictions.
    \item Measure the success of the team selection algorithm by analysing its ability to stay within the top 10\% of the global leaderboard during simulations.
    \item Benchmark the system’s results against popular AI-powered FPL platforms by comparing overall ranks, total points, and average weekly points.
    \item Compare data sources by conducting a statistical comparison against models using sentiment derived from other sources such as news articles.
    \item Ensure sufficient long-term prediction ability by analysing system performance across multiple gameweeks, accounting for variability in player form, injuries, and match conditions.
\end{itemize}
% -------------------------------------------------------------
% LITERATURE REVIEW
% -------------------------------------------------------------
\chapter[Literature Review]{Literature Review}
Due to the huge popularity of not only Fantasy Premier League, but the multiple games available for different sports under the fantasy sports genre, there exists a wide plethora of research and discussion around the given problem of maximising performance in fantasy sports. This study will look at the existing models that have been created and review how different machine learning methods stack up against each other. The existing research suggests that whilst many different techniques have been applied to varying degrees of success, using Twitter sentiment to predict player performance over the course of a whole season is a novel approach to the fantasy sports optimisation problem. The theory of wisdom of crowds and human-feedback aided models are explored to verify the benefits of using Twitter as a source of information. Finally, existing sentiment analysis methods are analysed to see how natural language is processed and evaluated to gather an overall sentiment value. The combination of all the research gives a good understanding for the necessity and validity of the approach this study focuses on. 
% -------------------------------------------------------------
% FANTASY SPORTS FORECASTING
% -------------------------------------------------------------
\section[Fantasy Sports Forecasting]{Fantasy Sports Forecasting}
The problem of optimising a player's score over the course of a whole fantasy season is an extremely difficult task. The magnitude of this difficulty was illustrated by Kristiansen et al. who developed a mathematical model to describe fantasy premier league, and found that during the 2017/18 season the top manager in the world only managed to achieve a score equivalent to 51.75\% of the optimal solution \cite{kristiansen}. Furthermore, the mean gap between the optimal solution and the top manager was around 60 points per week, whilst the mean gap between the top manager and the average manager was only 20 points per week. This huge gap highlights the complex intricacies of fantasy sports optimisation, where the sheer number of variables like player mood, injury status, and team rotation make it almost impossible for human strategies to get anywhere close to an optimal solution. The large discrepancy between optimality and the current peak of human performance shows the potential for advanced machine learning methods to bring new insights and strategies that could help bridge the gap between human strategies and optimal performance.

In terms of existing mathematical and machine learning models, many solutions for predicting performance have already been created by researchers dating back to the early 2000s. These models all largely focus on using historical data with objective, performance-based metrics that describe exactly what a player or team has done in the past few weeks, combined with some sort of score that represents how difficult their upcoming games are going to be. In 2012 Matthews et al. modelled the problem of choosing a team as a belief-state Markov decision process, where player selections were actions that had some reward value \cite{matthews}. Using a Bayesian Q-learning algorithm actions were chosen to maximise long-term reward, resulting in a team expected to score the most points. This approach retrospectively would have ranked within the top 1\% of all managers during the 2010/11 season of FPL.

In 2018 GS created a binary classification model to classify players into groups that are 1 - expected to score at least 4 points in the next gameweek, and 2 - expected to score less than 4 points \cite{raghunandh}. He experimented with using different tree models but eventually concluded using a Gradient Boost model was the most accurate, using historical performance data to generate a number between 0 and 1 for each player. A threshold value was then determined to best split the players into the two separate categories. This approach proved successful, with the model achieving a precision of 83\% when choosing players expected to score at least 4 points in the next gameweek.

In 2022 Rajesh et al. used random forests and gradient boosting machines to create a system enabling the "average interested person" to make better decisions about who to include in their teams \cite{rajesh}. A key feature found in this study was that training multiple models for each playing position (goalkeeper, striker etc.) drastically increased performance in comparison with using one model for each position. All models used in this study were found to outperform the average player by at least 20\%, with Gradient Boosting Machines (GBMs) performing the best.

Also In 2022 Bangdiwala et al. compared three different methods - Linear Regression, Decision Trees, and Random Forests - for predicting the total number of points players would score in their upcoming match \cite{bangdiwala}. He found that over the course of a whole season, the Linear Regression model performed the best with a smaller root mean square error and mean absolute error than the other two models. The Random Forest model was a close second and the Decision Tree model performed the worst. Another study in 2024 by Papageorgiou et al. compared 14 models for fantasy basketball and found the most effective models to be Random Forests, Bayesian Ridge and AdaBoost \cite{papageorgiou}. 
% -------------------------------------------------------------
% WISDOM OF CROWDS
% -------------------------------------------------------------
\section[Wisdom of Crowds]{Wisdom of Crowds}
In his 2005 book \textit{The Wisdom of Crowds}, James Surowiecki explains how collective groups are often much better at predicting things than individuals \cite{wisdom-of-crowds}. Aristotle is credited as the first person to write about this theory in his work \textit{Politics}. He stated "it is possible that the many, though not individually good men, yet when they come together may be better, not individually but collectively, than those who are so, just as public dinners to which many contribute are better than those supplied at one man's cost" \cite{politics}. This is illustrated in Francis Galton's \textit{Vox Populi} where he describes a country fair in Plymouth in 1906 \cite{galton}. During the fair, a contest was being held to guess the weight of an ox. Galton observed that the median guess of a crowd of 800 people was within 1\% accuracy of the correct number. Suroweicki uses this example to explain how "a crowd's individual judgement can be modelled as a probability distribution of responses, with the median value being close to the true value of the predicted quantity"\cite{wisdom-of-crowds}.

An expansion on this theory is a new technique dubbed "surprisingly popular" \cite{surprisingly-popular}. This technique was discovered during a study at MIT's Sloan Neuroeconomics Lab in collaboration with Princeton University. In the study, participants were asked a series of questions for which they had to provide what they thought was the correct answer, alongside what they thought the most popular answer would be. Researchers found that taking the average difference between the two responses as the correct answer, reduced errors by 21.3 in comparison to simple majority results, and 24.2 percent in comparison to confidence weighted results, where participants gave a confidence score alongside their answer to express how confident they were with their answer.

Taking this knowledge back into an FPL context, research exists detailing how the use of crowd-based metrics such as a player's ownership percentage among other mangers can benefit overall performance. In the earlier mentioned model built by GS, he states how using the concept of wisdom of crowds by adding features like player ownership\%, transfers in, and transfers out to his gradient boost model improved the precision from 75\% to 83.33\% \cite{raghunandh}. Another study in 2019 by Bonello et al. details how expanding on existing models with human feedback such as news articles and betting markets led to a performance increase of over 300 points over the course of a whole season, ranking within the top 0.5\% of players in the world \cite{bonello}. This is compared to a standard statistical model that only placed within the top 13\%. The Bonello study also details how they explicitly decided against the use of Twitter posts in their model, as they found it hard to accurately derive sentiment from tweets due to grammatical errors, emoji usage, and football specific jargon. However, this study aims to use proper pre-processing and more accurate sentiment analysis methods that have emerged since 2019 (detailed in subsequent chapters) to eliminate these concerns, as there is a huge amount of data available on Twitter that should not be overlooked and it offers a large crowd of people all sharing opinions.

This is backed up by a 2022 study where Whittaker found that 72\% of FPL players found social media to be at least somewhat influential in their decision making when it comes to their FPL team \cite{whittaker}. Twitter was also central to a 2019 study by Bhatt et al. where crowd wisdom was put to the test using Twitter sentiment as a metric for creating diverse crowds \cite{bhatt}. FPL related tweets were collected, and then matched to real people's FPL accounts. User's historical player selection was then collected, focusing specifically on who they chose to captain each week. The Twitter users were then clustered into diverse crowds based on the semantic diversity of their tweets, and further sampled from the clusters to create a final set of diverse crowds. Analysing the captaincy choice from these groups found that on average, the captaincy choice from a random group of 6 outperformed 73\% of individuals, and the choice of a diverse group of 6 outperformed 93\% of people.
% -------------------------------------------------------------
% SENTIMENT ANALYSIS
% -------------------------------------------------------------
\section[Sentiment Analysis]{Sentiment Analysis}
The desire to capture and give meaning to public opinions has long been seen throughout history, with democratic voting as a measure of public opinion first appearing in early Greek civilisation in the 5th century BCE \cite{athenian-democracy}. A paper by Droba published in the early 20th century outlines methods for collecting public opinion, explaining how early questionnaires were first deployed \cite{droba}. With the emergence of the internet in the last few decades, methods of gathering public opinion have shifted online making it much more efficient for organisations to gather relevant information. Companies have become interested in gathering opinions about their products or services through online reviews. With the explosion in popularity of social media, individual researchers have been able to gather information for tasks like predicting elections, stock market trends, and natural disasters \cite{mantyla}.

Liu defines sentiment analysis or opinion mining as the field of study that analyses people's opinions, sentiments, evaluations, attitudes, and emotions from written language \cite{liu}. Patel et al. adds to this definition explaining it is a type of classification in which machine learning techniques are used to identify positive and negative words or reviews in text-driven databases \cite{patel}. Shah outlines the different methods that can be used for sentiment analysis, explaining their pros and cons \cite{shah}. In her article she explains the different machine learning models that are commonly used like the Naive Bayes algorithm that uses probability to determine whether a piece of text should be classified as positive, negative, or neutral. Recurrent Neural Networks (RNNs) and their variants Long Short-term Memory (LSTM) are mentioned due to their ability to handle long term dependencies often found in natural language.

In 2019, a new language representation model called Bidirectional Encoder Representation from Transformers (BERT) was introduced by Devlin et al. with a focus on pre-training deep bidirectional representations by jointly conditioning on both left and right context in all layers \cite{BERT}. The ability of BERT to understand language context from both directions (unlike LSTM) more accurately meant it set new benchmarks in the natural language processing community, and has become a cornerstone of modern NLP research. BERT can be fine tuned on a multitude of tasks including sentiment analysis, and a study by Elankath et al. found that when used for sentiment analysis of Malayalam tweets, BERT was the top performing model in terms of accuracy when compared to other models like LSTM \cite{elankath}.

Targeted Aspect-Based Sentiment Analysis (TABSA) aims to determine sentiment toward specific targets, such as individuals or entities \cite{tabsa}. This idea is particularly important in the context of FPL sentiment analysis because tweets often mention multiple players. Identifying sentiment toward specific players is far more valuable than assessing the overall sentiment of a tweet. Sun et al. proposed a methodology using BERT to address TABSA by framing it as a sentence-pair classification task \cite{sun}. Their approach involves constructing an auxiliary sentence, such as "What do you think of the safety of location - 1?", and pairing it with the relevant context to identify the sentiment toward the specific target. Building on this idea, Hoang et al. advanced the methodology in 2019, achieving state-of-the-art results across various sentiment analysis benchmarks \cite{hoang}.
% -------------------------------------------------------------
% SUMMARY
% -------------------------------------------------------------
\section[Summary]{Summary}
From the research explored, it is clear that there is a large gap between the points scored by the top performing manager in the world and the ceiling of potential points available. This gap highlights the potentially undiscovered strategies that machine learning models could discover to help bridge that gap and outperform other human managers. The existing models used for optimising FPL performance have been able to perform well with some ranking inside the top 1\% of all players in the world. Of these models, some of the best performing include Random Forests, Linear Regression and Gradient Boosting Machines. It is also clear that incorporating human feedback and crowd wisdom into these models helps to improve their performance by a significant margin. Using high-performing deep learning models like BERT can give accurate sentiment analysis of tweets, whilst targeted aspect-based sentiment analysis can further improve this by capturing sentiment towards people and groups.
% -------------------------------------------------------------
% METHODOLOGY
% ------------------------------------------------------------- 
\chapter[Methodology]{Methodology}
% -------------------------------------------------------------
% OVERVIEW OF THE SYSTEM
% -------------------------------------------------------------
\section[Overview of the System]{Overview of the System}
The main system is comprised of three components: the sentiment analysis model, the expected points model, and the team selection algorithm. Developed in Google Colab \cite{colab} using Python, the sentiment analysis model is used to generate an overall sentiment held by twitter users towards a certain Premier League player ahead of an FPL gameweek. This sentiment score is incorporated into an existing dataset, containing historical data for FPL players to enhance a point prediction model. Whilst many expected points models for FPL points predictions have already been developed, none have leveraged twitter sentiment. By adding this feature, it is hoped model performance will increase due to the added knowledge and crowd wisdom the twitter data contains. 

The expected points model outputs a final prediction for each player, representing its guess for how many FPL points that player will score in the upcoming gameweek. A team selection algorithm will take this information and identify the squad expected to maximise the number of points scored, whilst satisfying all budgetary, positional, and team constraints.

The sentiment analysis model is trained using a publicly available dataset of Premier League related tweets from the 2022/23 season. A fine-tuned BERT model is used to classify the data and then sentiment scores are aggregated to produce a final sentiment value for each gameweek. Respective historical data from the same time period is then taken from a community-run github repository, and combined with the sentiment values for each player. A gradient boosting machine (GBM) is trained on the enhanced dataset, producing a final expected points value for each player.

Finally, a team selection algorithm takes the expected points values, and selects an optimal squad of 15 players that satisfies the FPL team constraints. This problem is formulated as an optimisation problem with the objective being to maximise total expected points within a given budget. Integer linear programming is used to find the best possible team composition.

Together these components form an intelligent support system for FPL managers that incorporates crowd wisdom as well as historical data. The final system should help managers to make better and more informed decisions that lead to an increase in their overall rank.
% -------------------------------------------------------------
% SENTIMENT ANALYSIS MODEL
% -------------------------------------------------------------
\section[Sentiment Analysis Model]{Sentiment Analysis Model}
To capture crowd wisdom through twitter, a sentiment analysis model is needed to automatically label tweets and gauge users opinions towards each player. Whilst it is possible to train a sentiment analysis model from scratch, this study uses a pre-trained BERT model \cite{BERT} available for free from Google. This approach leverages access to a model that has been trained on billions of texts, giving it an extremely complex understanding of natural language. A model trained from scratch could not hope to replicate such complexity given the timeframe of this project. It would also be unreasonable to expect such a large task to be completed without the use of advanced computing machinery unavailable to individual researchers without the backing of a large corporation. The BERT model has been chosen specifically because it has been proven to outperform other state-of-the-art models like LSTM in sentiment analysis tasks\cite{elankath, hoang}.

This study will use fine-tuning to refine BERT for the specific task of sentiment analysis on FPL tweets. Fine tuning is the process of adapting a pre-trained model for specific tasks or use-cases \cite{fine-tuning}. This is a much less computationally demanding task than fully training a model and requires only a small to mid sized dataset. It also uses less computing power, ideal for a project of this scale. By training the extensive natural language knowledge that BERT already has on a suitable dataset of FPL tweets, a sophisticated and accurate sentiment analysis model that suits the needs of the project should be achievable.

\subsection[Data Collection]{Data Collection}
Identifying and collecting suitable data for fine-tuning the model is a critical step in creating a model with optimal performance. The quality of the fine-tuning data directly affects the model's output quality \cite{gigo}, so it was important to chose a dataset carefully.

The first approach considered was manual data collection directly via Twitter, however, exploring this approach led to many dead ends. The first problem was the financial cost of using the official Twitter API, which has seen a price increase from \$3 per month to \$100 per month since the company was sold in 2022 \cite{twitter-costs}. This change has led to many free tools aimed at helping researchers leverage twitter data for their studies being shut down, impacting the entire research field by limiting the options they have when it comes to collecting data. Without being able to use these tools, and with the official API unaffordable, using the API to efficiently collect data was not an option. Whilst other methods of collecting data exist, such as writing code scripts to scrape data from the Twitter website, or using 3rd party browser extensions, these methods are against the Twitter terms of service and could result in a permanent account ban if caught.

Manual collection of data on Twitter is an acceptable approach within the terms of service, however it is an extremely tedious process. When considering a dataset needed to fine tune a BERT model should contain thousands of entries, as well as the time constraints of this project, manually collecting and then labelling thousands of tweets was also not an option. 

The chosen approach was to use an existing dataset published on popular machine learning forums such as Hugging Face and Kaggle. These sites have thousands of active users contributing to a wide range of research topics. As such, finding a dataset for a similar purpose to this study's was not difficult. 4 potential datasets were identified on Kaggle, and include:

\begin{itemize}
    \item \href{https://www.kaggle.com/datasets/tirendazacademy/fifa-world-cup-2022-tweets}{Fifa World Cup 2022 Tweets with Sentiment Labels} - (30k tweets)
    \item \href{https://www.kaggle.com/datasets/wjia26/epl-teams-twitter-sentiment-dataset}{Premier League Teams Tweets with Sentiment Labels} - (460k tweets)
    \item \href{https://www.kaggle.com/datasets/prasad22/fpl-tweets-dataset}{FPL Tweets from 2012 - 2023, unlabelled} - (110k tweets)
    \item \href{https://www.kaggle.com/datasets/firefly55lm/tweets-footballers-premier-league}{Premier League Players Tweets with Sentiment Labels} - (167k tweets)
\end{itemize}


A further inspection of these 4 datasets was carried out to identify the most suitable given the needs of the model being built. 

The first dataset, whilst labelled, only contained tweets for the first day of the world cup. This meant a majority of tweets were centered around the event as a whole and the controversy of Qatar hosting, with little mention of players or actions they had carried out during football matches. The second dataset was much more suited to the needs of the model, however had the drawback of being centred around teams rather than individual players. The third dataset was focused specifically on FPL which appeared ideal at first. However it was found that this dataset contained a lot of unrelated tweets from people tweeting about their own performance in FPL, instead of expected performance of actual players.
	% -------------------------------------------------------------
	% NOISY DATA
	% -------------------------------------------------------------
	\begin{figure}[htbp]
	\centering
	\includegraphics[scale=0.5]{fpl-tweets-dataset}
	\caption{Examples of noisy tweets irrelevant to player performance, such as personal FPL commentary}
	\label{default}
	\end{figure}
	% -------------------------------------------------------------
	% NOISY DATA
	% -------------------------------------------------------------

The fourth and final dataset was identified as the most suitable for this study. This dataset contains a far greater number of relevant tweets, focused on individual Premier League players from the 2022/23 season (it is almost impossible to find more recent data, due to the Twitter API price changes coming into effect from 2023). As well as the relevant focus of the tweets themselves, this dataset contains player labels for each row which is perfect for construction of an auxiliary sentence for target-based sentiment analysis. Instead of manually searching for players in each tweet, the existing \textit{player\_name} column in the dataset can be used. Finally, this dataset comes with the advantage of being almost fully cleaned of any noise including emojis, links, images, etc. reducing the amount of pre-processing required.

The sentiment labels provided in the dataset were automatically created by the Vader \cite{vader}, and TextBlob \cite{textblob} models. These models are fairly accurate however it is worth noting they do incorrectly label some tweets and therefore this dataset is not perfect. A potential solution to this problem is manually labelling the tweets, however this would require some criteria to match certain labels to tweets, and would also be extremely time consuming. Given the time constraints of this project and the limited human resources available for labelling, the dataset's size would have to decrease significantly to use a manual labelling approach. It is hoped that using the existing labels will be more beneficial due to the much larger dataset size available with this approach, which should result in a more accurate final model.
\newpage
% -------------------------------------------------------------
% DATA PRE-PROCESSING
% -------------------------------------------------------------
\subsection[Data Pre-Processing]{Data Pre-Processing}
The dataset was cut down to a more suitable size of around 20k tweets to keep training time reasonable. This was done using a random sampling approach to reduce biases. Reducing the dataset size will most likely affect the models performance negatively, however given the time constraints of this project the computation time required to train the model on a dataset over 100k rows would have been too large.

Many tweets in the dataset mention multiple players, with some even containing conflicting sentiments e.g. "Haaland was poor today, massively outshone by Salah". In order to capture the separate sentiment values (target based sentiment) the method introduced by Sun et al. \cite{sun} was used, where an auxiliary sentence was created posing a question to the model. The chosen format was: "What do you think of the sentiment towards [player]?". Using this approach, the model inputs and outputs will look like:  

\begin{itemize}
    \item \textbf{Input:} What do you think of the sentiment towards Haaland?  
    \item \textbf{Output:} Negative
    \item \textbf{Input:} What do you think of the sentiment towards Salah?  
    \item \textbf{Output:} Positive
\end{itemize}

While a more FPL-specific auxiliary sentence (e.g., 'Should [player] be in my FPL squad?') was considered, it risked introducing noise due to sentiment labels from Vader/TextBlob. For example, 'Jamie Vardy has great hair!' might be labeled as 'yes' (suggesting he should be in an FPL squad) despite being unrelated to performance. Due to this potential issue, a more generic question was used with the intuition being that positive overall sentiment should closely correlate with good FPL performance.
 
After adding the auxiliary sentence to the dataset, the rest of the tweets were cleaned of any noise using regular expressions to remove links, hashtags, and other random characters. Duplicates and empty rows were removed, and any other unnecessary columns were dropped. Removing noise from tweets ensures the model learns only relevant information, ultimately improving performance.

\subsection[Fine-Tuning BERT]{Fine-Tuning BERT}

% -------------------------------------------------------------
% EXPECTED POINTS MODEL
% -------------------------------------------------------------
\section[Expected Points Model]{Expected Points Model}
\subsection[Data Collection and Engineering]{Data Collection and Engineering}
\subsection[Gradient Boosting Model]{Gradient Boosting Model}
\subsection[Model Training and Evaluation]{Model Training and Evaluation}
% -------------------------------------------------------------
% OPTIMAL TEAM SELECTION ALGORITHM
% -------------------------------------------------------------
\section[Optimal Team Selection Algorithm]{Optimal Team Selection Algorithm}
\subsection[Constraints]{Constraints}
\subsection[Algorithm]{Algorithm}
% -------------------------------------------------------------
% SUMMARY
% -------------------------------------------------------------
\section[Summary]{Summary}
% -------------------------------------------------------------
% SENTIMENT ANALYSIS MODEL
% -------------------------------------------------------------
\chapter[Sentiment Analysis Model]{Sentiment Analysis Model}
\section[The BERT Model]{The BERT Model}
Bidirectional Encoder Representation from Transformers or BERT \cite{BERT}, is a pre-trained language representation model based on the transformer architecture proposed in the now infamous 2017 paper "Attention is All You Need" \cite{attention}. The transformer is a deep learning architecture that is the backbone of some of the most prominent achievements in AI in recent years, including OpenAI's ChatGPT \cite{chatGPT}.

        % -------------------------------------------------------------
	% TRANSFORMER ARCHITECTURE
	% -------------------------------------------------------------
	\begin{figure}[htbp]
	\begin{center}
	\includegraphics[scale=0.35]{transformer-architecture}
	\caption{Architecture of a Transformer \cite{attention}}
	\label{default}
	\end{center}
	\end{figure}
	% -------------------------------------------------------------
	% TRANSFORMER ARCHITECTURE
	% -------------------------------------------------------------

The original transformer architecture was designed for language translation, but has since been adopted and used for a vast array of other tasks, including for pre-trained systems like BERT and Generative Pre-trained Transformers (GPTs) \cite{transformer-applications}. The transformer is made up of two parts; the encoder, and the decoder. A few months after the transformer architecture was proposed, researchers started experimenting with the idea of separating the encoder and the decoder resulting in some incredible breakthroughs. The creation of encoder-only and decoder-only transformers is what has resulted in the AI boom that we have seen today with large language models like GPTs \cite{chatGPT} and BERT \cite{BERT}. BERT is made up of deeply bidirectional encoder-only transformers which although have been more understated than their decoder-only siblings (used in ChatGPT), are still extremely powerful for natural language tasks like sentiment analysis.

\subsection[Tokenisation]{Tokenisation}
To split large pieces of text into more manageable chunks of data so that natural language models can process and understand them better, a process known as tokenisation is carried out on the training data \cite{jurafsky-speech}.

In the BERT model, raw text is broken down into tokens using the WordPiece tokenisation algorithm \cite{word-piece}. As opposed to word-level or character-level tokenisation, where whole words or individual characters represent tokens, the WordPiece algorithm is a subword-level tokenisation algorithm. This means words are split into one or more tokens such as \texttt{'token'} and \texttt{'isation'}.

The algorithm starts by generating an initial vocabulary from the training data, by splitting each word in the data into individual characters. Each character that does not start a word is prefixed with \texttt{'\#\#'} to indicate it is a sub-word, so the word 'word' would be split like this: \texttt{'w' '\#\#o' '\#\#r' '\#\#d'}. WordPiece then computes a score for each pair that occurs in the training data using the following formula:

\[
\text{score} = \frac{\texttt{pair\_freq}}{\texttt{first\_element\_freq} \times \texttt{second\_element\_freq}}
\]

The pair with the highest score is merged into 1 token and added to the vocabulary, and the same merge is applied to the set of pairs of tokens. This process is then repeated until the vocabulary reaches a desired size. The BERT vocabulary is sized around 30k tokens \cite{BERT}.

Once the vocabulary has been generated, words can be tokenised. This is done by iterating through each word in a piece of text, and looking for the longest available token at the start of a word. If one is found, the algorithm does the same for the next part of the word, and so on until the whole word is tokenised. If the algorithm finds a part of a word that has no matching token in the vocabulary, the \textbf{entire word} is given the special token [UNK] or unknown \cite{word-piece}. So if you had a vocabulary of ['hel', 'lo', 'worl'] and the sentence 'hello world', the resulting tokenisation output would be ['hel', 'lo', [UNK]].

The BERT model also uses the special tokens [CLS] (classifier) and [SEP] (separator). The CLS token is placed at the very start of the input, and for sentiment analysis tasks it has a hidden state associated with it that will be passed to a classifier to predict sentiment after the input has been processed \cite{fine-tune-bert}. The SEP token is used to separate two sentences for tasks such as question-answering and dual-sentence classification \cite{BERT}.

\newpage
\subsection[Input Embeddings]{Input Embeddings}
After text has been tokenised, it needs to be converted into numerical values using text encoding. Text encoding allows raw text to be handled by neural networks which can only take numerical values as input \cite{jurafsky-speech}. The raw text is converted into a set of vectors called word embeddings, which can be processed by the encoder's neural network. Embeddings give meaning to tokens, with similar tokens like 'great' and 'awesome' closer together in vector space, and opposites like 'sad' and 'happy' further apart \cite{mikolov-word2vec}.

The embedding layer creates word embeddings (or subword embeddings, in the case of BERT) for each token in the input using 3 different types of embeddings: token embeddings, positional embeddings, and token type embeddings \cite{embedding-layer-diagram}. 

\begin{itemize}
    \item \textbf{Token embeddings} are vectors that have been learned during the model's pre-training phase to place similar tokens close together in vector space, they are stored in an embedding matrix which maps each token to a specific embedding. 
    \item \textbf{Positional embeddings} are also learned vectors given to each token that represent their positions in the input sentence.
    \item \textbf{Token type embeddings} are often used for two-sentence NLP tasks like question-answering to identify which sentence a token belongs to. For one-sentence tasks like next word prediction, all tokens are assigned the same vector.
\end{itemize}

The embedding layer then sums these embeddings together, and applies normalisation to their sum. The resulting vector output contains meaningful information about each token and its position in the input. These embeddings are passed into the subsequent transformer layers of the BERT model for processing.
        % -------------------------------------------------------------
	% EMBEDDING LAYER DIAGRAM
	% -------------------------------------------------------------
	\begin{figure}[htbp]
	\begin{center}
	\includegraphics[scale=0.5]{bert-embedding-layer}
	\caption{BERT Embedding Layer \cite{embedding-layer-diagram}}
	\label{default}
	\end{center}
	\end{figure}
	% -------------------------------------------------------------
	% EMBEDDING LAYER DIAGRAM
	% -------------------------------------------------------------
\subsection[Multi-Head Attention]{Multi-Head Attention}
A fundamental concept of the transformer architecture is 'self-attention'\cite{attention}. Given the sentence: \textit{'I took the pizza out of the oven and then ate it.'} it is apparent to any human that 'it' refers to the pizza and not oven, based on the surrounding context. Neural networks however lack the ability to identify this relationship, and require mechanisms like self-attention in order to resolve such ambiguities.

Self-attention allows neural networks to identify the importance of each token in relation to the other tokens in the input. To calculate attention values, BERT uses multiple attention 'heads' which all focus on different linguistic features of the input. Each head involves 3 components: the query (Q), the key (K), and the value (V) matrices \cite{attention}. Each of these matrices are made up of the input embeddings of the previous layer (a vector for each token), and are identical to ensure that every token in the input interacts with every other token, including itself (the idea of self-attention).
	% -------------------------------------------------------------
	% DOT PRODUCT ATTENTION
	% -------------------------------------------------------------
	\begin{figure}[htbp]
	\begin{center}
	\includegraphics[scale=0.4]{dot-product-attention}
	\caption{Scaled Dot Product Attention \cite{attention}}
	\label{default}
	\end{center}
	\end{figure}
	% -------------------------------------------------------------
	% DOT PRODUCT ATTENTION
	% -------------------------------------------------------------
\[
\text{Attention}(Q, K, V) = \mathrm{softmax}\left(\frac{QK^T}{\sqrt{d_k}}\right) V
\]

Here $QK^T$ computes the similarity between the key and query matrices, scaled by $\sqrt{d_k}$ (the dimension of the embedding vectors), and then the softmax function \cite{softmax} normalises these values into attention weights. These weights are applied to the value matrix, resulting in an attention net matrix representing the contextual relationship between tokens.

\newpage
For all attention layer heads (BERT uses $h$=12 attention heads \cite{BERT}), there is a linear layer that uses learned weights to project the input embeddings into separate Q, K, and V matrices for each head. This way different linguistic features can be explored by the model to improve its learning ability. The outputs (attention nets) from each head are then concatenated to form a unified representation:
	% -------------------------------------------------------------
	% MULTI-HEAD ATTENTION
	% -------------------------------------------------------------
	\begin{figure}[htbp]
	\begin{center}
	\includegraphics[scale=0.4]{multi-head-attention}
	\caption{Multi-Head Attention \cite{attention}}
	\label{default}
	\end{center}
	\end{figure}
	% -------------------------------------------------------------
	% MULTI-HEAD ATTENTION
	% -------------------------------------------------------------
\[
\begin{split}
\text{MultiHead(Q, K, V)} = \mathrm{Concat}\left(\text{head}_1, \ldots, \text{head}_h\right) W^O \\
\text{where } \text{head}_i = \mathrm{Attention}\left(QW_i^Q, KW_i^K, VW_i^V\right).
\end{split}
\]

Here, $W_{i}^Q, W_{i}^K , W_{i}^V$ are the learned projection matrices for the $i$-th head, and $W^O$ is the projection matrix for the final linear layer. This final layer aggregates the information from all attention heads to produce a final self-attention representation for the input \cite{attention}.

\newpage
\subsection[Feed-Forward Network]{Feed-Forward Network}
The feed-forward neural network (FFNN) is the final stage of the encoder. It is a type of multi-layer perceptron (MLP), used for transforming the output of the self-attention mechanism into a refined representation that can capture deeper relationships within the input sequence \cite{ffnn}.

The FFNN processes input embeddings by first projecting them into a higher-dimensional space. Specifically, the 768-dimensional input for each BERT token embedding \cite{BERT} is transformed into a 3072-dimensional space via a linear projection. This expansion allows the model to explore more complex interactions among the input features. Following this, a ReLU activation function is applied to allow the model to identify more intricate non-linear patterns in the language. Finally, another linear projection reduces the representation back to its original dimensionality, ensuring compatibility with the encoder's output structure \cite{ffnn}.

The FFNN is essential for capturing patterns and interactions that the self-attention mechanism alone cannot model. Its reliance on simple matrix operations makes it computationally efficient and highly parallelisable, leveraging modern hardware accelerators like GPUs and multi-core CPUs to significantly reduce training time. To ensure stable training and enhance convergence, the FFNN's output undergoes normalisation, producing the final output representation for the encoder.\\

\subsection[Final BERT Architecture]{Final BERT Architecture}
The previous sub-sections have explored the components of the encoder, but the BERT model actually makes use of multiple encoders stacked together (12 for BERT$_{\text{base}}$ which this study uses) \cite{BERT}. This is what makes BERT a deep-learning model, as it incorporates many layers into its structure. The stacking of multiple encoders allows the model to progressively learn more high-level representations of the input data as it passes through the layers.

Each encoder layer outputs a 768-dimensional matrix capturing increasingly complex patterns and relationships in the data at every subsequent encoder layer. The output of the final encoder layer represents the most refined and meaningful representations for each token \cite{BERT}, which can then be used for downstream tasks. For sentence classification tasks like sentiment analysis, the special token ([CLS]) is used, and its final vector representation serves as the input to the classification head. The classification head takes the input vector and outputs a score representing the model's sentiment prediction \cite{fine-tune-bert}.

The depth of the architecture, combined with the self-attention mechanism, enables BERT to model long-range dependencies effectively. This is a huge benefit for processing natural language, where relationships between words are often spread across entire sentences or paragraphs.

\subsection[Pre-Training]{Pre-Training}
BERT was pre-trained using two tasks: Next Sentence Prediction (NSP) and Masked Language Modelling (MLM) \cite{BERT}.

The first task, MLM, is a technique for training deeply bidirectional models like BERT. Typical language models process tokens sequentially left-to-right, using preceding tokens in a sentence to predict the next. BERT, however, processes all tokens in a sentence simultaneously and can see both preceding and following tokens for each word. To predict a missing token in the middle of a sentence, BERT could "cheat" by leveraging full context, including the token to predict. To address this, word "masks" are applied to randomly hide tokens in a sentence \cite{BERT}, tasking BERT to predict them using only surrounding context. This technique enables BERT to learn bidirectional relationships, resulting in a more robust model than typical left-to-right ones.

The second task, NSP, teaches BERT to understand relationships between two sentences, useful for downstream tasks like Question Answering (QA) and Natural Language Inference (NLI). In this task, BERT is fed two input sentences, A and B, and must guess whether sentence B follows A or is unrelated. BERT is trained on an even split of positive and negative cases. This simple task produces remarkable results, with the final model achieving 97\% accuracy on NSP tasks \cite{BERT}.

By leveraging these tasks across millions of sentences and performing millions of optimisation steps, BERT learns complex relationships between words and sentences \cite{BERT}. The model's weights, including those for input embeddings and self-attention mechanisms, are updated via backpropagation \cite{backpropagation} during this process.

	% -------------------------------------------------------------
	% BERT PRE-TRAINING
	% -------------------------------------------------------------
	\begin{figure}[htbp]
	\begin{center}
	\includegraphics[scale=0.3]{bert-pre-training}
	\caption{BERT Pre-Training \cite{BERT}}
	\label{default}
	\end{center}
	\end{figure}
	% -------------------------------------------------------------
	% BERT PRE-TRAINING
	% -------------------------------------------------------------
	
This extensive pre-training enables BERT to capture nuanced patterns and relationships in text, which are then fine-tuned on downstream tasks like sentiment analysis, question answering, and named entity recognition \cite{fine-tune-bert}.
\subsection[BERT fine-tuning]{BERT fine-tuning}
Fine-tuning is the process of adapting a pre-trained model to a specific task by training it further on task-specific data \cite{fine-tuning}. For sentiment analysis, this involves adding a task-specific classification layer on top of the BERT encoder stack \cite{BERT}. Fine-tuning combines BERT's pre-trained understanding of language with additional knowledge it needs for the unique challenges of sentiment analysis.

After an input sentence passes through the model, the final hidden state vector $\mathbf{h}$ corresponding to [CLS] acts as a condensed representation of the entire sequence, capturing its sentiment information. A softmax \cite{softmax} classifier is then applied to predict the probability of each sentiment label $c$ \cite{fine-tune-bert}:
\[
p(c|\mathbf {h}) =\mathrm {softmax}(W\mathbf {h})
\]
Here, $W$ represents the task-specific parameter weights matrix, and the sentiment label with the highest probability is chosen as the final prediction.
	% -------------------------------------------------------------
	% BERT FINE-TUNING
	% -------------------------------------------------------------
	\begin{figure}[htbp]
	\begin{center}
	\includegraphics[scale=0.35]{sentiment-diagram}
	\caption{Single sentence classification \cite{sentiment-cls-diagram}}
	\label{default}
	\end{center}
	\end{figure}
	% -------------------------------------------------------------
	% BERT FINE-TUNING
	% -------------------------------------------------------------

Using a labeled dataset of texts and their corresponding sentiment labels (e.g., ‘positive,’ ‘negative,’ ‘neutral’), the model is trained to minimise a classification loss, such as cross-entropy. During training, the model's predictions are compared to their labels, and all weights (including embedding, self-attention, classification etc.) are updated iteratively to improve accuracy. This process is repeated for a set number of epochs (complete passes through the dataset) to improve accuracy whilst preventing overfitting.

For the more complicated task of Targeted Aspect-Based Sentiment Analysis (TABSA), a proven fine-tuning approach involves reframing the problem as a sentence-pair classification task \cite{sun}. This method involves constructing an auxiliary sentence to specify the target and aspect BERT needs to predict. In the context of FPL, the auxiliary sentence could be: "Should \textit{player} be in my FPL team?". Then when given a sentence like "Mo Salah was excellent, but Haaland was awful," an auxiliary sentence is created for each player by replacing the placeholder \textit{player} with the actual player’s name;  "Should Mo Salah be in my FPL team?" and "Should Haaland be in my FPL team?".

During fine-tuning, BERT processes the auxiliary sentence followed by the [SEP] token and the original sentence for each player. Classification is done in the same way as normal sentiment analysis with the hidden vector $h$ used to predict the label (in this example the labels would be answers to the auxiliary question: 'yes', 'no' or 'unsure'). This approach enables BERT to focus on the sentiment directed toward each specific player, rather than providing a single sentiment value for the entire sentence. A single value would likely be uninformative, as positive and negative sentiments could cancel each other out. By targeting individual players, this method ensures more accurate predictions, with players labeled according to their specific sentiment.

	% -------------------------------------------------------------
	% BERT FINE-TUNING
	% -------------------------------------------------------------
	\begin{figure}[htbp]
	\begin{center}
	\includegraphics[scale=0.35]{sentence-pair-cls-diagram}
	\caption{Sentence-pair classification \cite{sentence-pair-cls-diagram}}
	\label{default}
	\end{center}
	\end{figure}
	% -------------------------------------------------------------
	% BERT FINE-TUNING
	% -------------------------------------------------------------

This fine-tuning process not only adapts BERT to specific targets but also allows it to capture domain-specific language features that differ from general usage. For instance, while the word “soft” might generally convey positive sentiment (e.g., a soft blanket), in the FPL context, calling a player "soft" is often a negative sentiment. By fine-tuning BERT on domain-specific tweets, the model can learn such nuanced language patterns, improving its accuracy in the target domain.

Fine-tuning BERT is computationally efficient compared to training a model from scratch and requires a relatively small amount of labeled data. This makes it a practical choice for sentiment analysis tasks across diverse domains.
\newpage
% -------------------------------------------------------------
% MODEL IMPLEMENTATION
% -------------------------------------------------------------
\newpage
\section[Model Implementation]{Model Implementation}
The BERT model used in this study to predict sentiment for FPL players was fine-tuned in Google Colab \cite{colab} using Python. Google Colab is a cloud computing platform that was chosen as it allows free GPU and TPU access, much better suited to fine-tuning a deep learning model than a CPU found in a typical laptop, helping reduce training time. Its notebook format also allows for simple and clear separation of functions, allowing for single blocks of code to be independently executed. This allows for easy debugging and troubleshooting of code, improving the development experience.
 
Python was chosen as the preferred language due to the extensive number of libraries that exist supporting the development of deep learning models. Some of the most popular include Hugging Face's Transformers library \cite{huggingface-transformers} which contains APIs for hundreds of pre-trained transformer models including BERT. The transformers library also includes tokenisers for each model and a PyTorch \cite{pytorch} Trainer class that allows fine-tuning to be done with just a few lines of code. The model's hyper-parameters can be fine tuned and optimised using the Optuna library \cite{optuna} to explore the best possible combinations. Use of these libraries ensure correct approaches are taken, and massively reduces development time by simplifying the entire process.

\subsection[Training the Model]{Training the Model}
The first step of training was to configure the TrainingArguments from the transformers library. This class contains the hyper-parameters for the model such as learning rate, batch size, epochs, and weight decay. These hyper-parameters can be tweaked to improve model performance and prevent overfitting during training. Initially, these were set to default values suggested by the Hugging Face tutorial, with the intention to optimise them properly later (see 4.2.4).  
	
After these arguments were configured, the next step was to set up the Trainer object. The pre-trained \texttt{BertForSequenceClassification} model was loaded in from the transformers library. This model is configured for classification tasks on 2 input sentences with a suitable classification head already implemented, exactly what is needed for target-based sentiment analysis with an auxiliary sentence. The specific model used was 'bert-base-cased' which is just the base variant of the BERT model that is case-sensitive. The Trainer was initialised by passing in the model, datasets and training arguments. Once the Trainer had been initialised, the training process was started by calling the \texttt{train()} method.

Training took around 90 minutes, and the resulting model scored an initial accuracy of around 90.34\%. Other performance metrics such as precision and F1 score were configured and will be discussed further during final model evaluation. Focusing on the validation loss, it is clear that the initial model is starting to overfit the training data as the epochs go on, evidenced by the increasing values after epoch 2. This is something that will be refined using hyper-parameter tuning.
	% -------------------------------------------------------------
	% INTIAL METRICS
	% -------------------------------------------------------------
	\begin{figure}[htbp]
	\centering
	\includegraphics[scale=0.52]{initial-train}
	\caption{Performance metrics for the initial model}
	\label{default}
	\end{figure}
	% -------------------------------------------------------------
	% INITIAL METRICS
	% -------------------------------------------------------------


Once trained, the model was tested to ensure the it worked as expected. 5 samples from the unseen data were taken and passed through the model to infer an output. The unseen data was processed by the same tokeniser function used for the training data to ensure the data used for model inference was pre-processed in the exact same way as the training data. This is crucial for achieving an accurate prediction from the model by ensuring the model recognises the inputs and can leverage knowledge learned during training to the fullest extent. Inference was done using the built in \texttt{eval()} method that the transformers library provides for every model.
	% -------------------------------------------------------------
	% INFERENCE
	% -------------------------------------------------------------
	\begin{figure}[htbp]
	\centering
	\includegraphics[scale=0.35]{inference}
	\caption{The model's predictions on unseen data}
	\label{default}
	\end{figure}
	% -------------------------------------------------------------
	% INFERENCE
	% -------------------------------------------------------------

Model outputs appear to be mostly correct and are easily retrieved through the built-in methods provided by the transformers library. This allows for a quick and efficient way to generate a list of sentiment predictions for a large set of tweets, necessary for the final system. 

\newpage
\subsection[Hyper-parameter Optimisation]{Hyper-parameter Optimisation}
Selecting a suitable set of hyper-parameters is crucial for model performance. Adjusting these values during training can help the model to avoid overfitting the training data, minimise validation loss, and maximise accuracy \cite{hyperparameters}. Typically when implementing a machine learning model, researchers manually tweak hyper-parameters and spend significant time exploring configurations that may not improve performance meaningfully. This manual process is inefficient and resource-intensive, so by using an automatic framework, less computational resources and time is spent developing the model.

The framework chosen was Optuna \cite{optuna}, an automatic hyper-parameter optimisation framework for python that finds the most optimal hyper-parameter values via trial and error. It conducts a series of trials and uses the previous trial to identify promising potential tweaks that could be made to improve performance. This process is repeated and a history of trials is kept throughout the process to guide the next changes made to the hyper-parameters. Through enough trial and error, optimal hyper-parameter values are found. Optuna is a well documented library with great Python integration, making it an ideal choice for this project.

The first step in configuring Optuna was to specify the hyper-parameters to be tweaked and the space to be explored. The search space for hyper-parameters was defined based on the values recommended in the original BERT paper \cite{BERT}. The chosen hyper-parameters along with their respective search space were:

\begin{itemize}
	\item Learning Rate ($2 \times 10^{-5}$ to $5 \times 10^{-5}$) - determines the step size at which the model updates its weights during training. A higher learning rate can speed up training time but risks overshooting the optimal parameters, whilst a lower learning rate ensures stability but may result in slower training.
	\item Batch Size (16 or 32) - Batch size refers to the number of training samples processed simultaneously before updating the model's weights. A smaller batch size can introduce noise in the gradient updates, which sometimes aids in escaping local minima but can lead to slower convergence. A larger batch size offers smoother gradient updates but may require more computational resources and risks overfitting.
	\item Number of Epochs (2 to 4) - specifies how many times the entire training dataset will be processed by the model during fine-tuning. Too many epochs can cause the model to focus on specific patterns in the training data, leading to overfitting. Too few epochs can limit the model's ability to learn from the data and lead to poor accuracy.
\end{itemize}

\newpage
An objective function was defined to guide the hyper-parameter tuning process. This function evaluates the performance of each 'trial' which is a model trained on a set of hyper-parameters suggested by Optuna. It defines the search space and hyper-parameters available, and then feeds Optuna's suggestions into each model's training arguments. It also specifies which metric to use to evaluate this models - validation loss in this case. The validation loss is a measure of how far away the model's predictions are from the labelled values. To achieve an accurate model, this metric should be as low as possible to indicate the model's guesses are very close to the truth.

The datasets used during this stage were slimmed down to reduce training time. Evaluating models trained with the entire dataset would have taken too long (90 mins per trial) and used too much computing resource, so instead a dataset 25\% of the size was used. This approach should give a general view of the best configurations, which can be explored further using the whole dataset.

An Optuna study was created using the \textit{create\_study} method, and then run for 10 trials. This amount should allow Optuna to explore a sufficient amount of configurations while keeping resource usage to a minimum. Because Optuna chooses each subsequent configuration based on the previous one, each new configuration should explore meaningful search space and therefore enable a near optimal configuration to be found in 10 trials.

The configurations and resulting validation loss for each trial were:

\begin{table}[h!]
\centering
\renewcommand{\arraystretch}{1.5}
\setlength{\tabcolsep}{8pt}
\begin{tabular}{|c|c|c|c|c|}
\hline
\textbf{Trial} & \textbf{Learning Rate} & \textbf{Batch Size} & \textbf{Epochs} & \textbf{Validation Loss} \\ \hline
1 & $3.895 \times 10^{-5}$ & 16 & 3 & 0.626 \\ \hline
2 & $3.996 \times 10^{-5}$ & 16 & 4 & 0.629 \\ \hline
3 & $3.946 \times 10^{-5}$ & 32 & 3 & 0.783 \\ \hline
4 & $2.464 \times 10^{-5}$ & 32 & 3 & 0.847 \\ \hline
5 & $4.504 \times 10^{-5}$ & 16 & 3 & \textbf{0.524} \\ \hline
6 & $2.609 \times 10^{-5}$ & 16 & 2 & 0.876 \\ \hline
7 & $3.742 \times 10^{-5}$& 32 & 2 & 0.890 \\ \hline
8 & $2.007 \times 10^{-5}$ & 16 & 2 & 0.844 \\ \hline
9 & $3.494 \times 10^{-5}$ & 32 & 3 & 0.736 \\ \hline
10 & $3.710 \times 10^{-5}$ & 16 & 3 & 0.705 \\ \hline
\end{tabular}
\caption{Hyper-parameter configurations with associated validation loss}
\label{tab:scoring_system}
\end{table}


\newpage
\subsection[Model Evaluation]{Model Evaluation}
The three best configurations discovered during hyper-parameter optimisation were used to fully train 3 models. It was initially discovered that these configurations started to overfit due to an increase in data samples, so to compensate for this training was reduced by one epoch. Whilst optimising the hyper-parameters with the full dataset would give the best results, the time taken to do this (around 15 hours for 10 trials) would have been too great, so this approach is chosen as a compromise.

These models were evaluated using the following performance metrics:

\begin{itemize} 
	\item \textbf{Accuracy} - The percentage of predictions that the model correctly makes across all classes. 
	\item \textbf{Precision} - The combined percentage of predictions that are correct for each class, weighted by the number of instances in each class. E.g. if 30 samples are labelled positive, and the model correctly predicts them all, but also incorrectly predicts 10 more samples as positive the precision for that class will be $30/40 = 75\%$. 
	\item \textbf{Recall} - The combined percentage of true instances that were correctly identified by the model, also weighted by the number of instances in each class. For the same example where all 30 positive samples were correctly predicted, the recall would be 100\% for the positive class.
	\item \textbf{F1 Score} - The harmonic mean of precision and recall, calculated as: 
 \[
F_1 = 2 \times \frac{\text{precision} \times \text{recall}}{\text{precision} + \text{recall}}
\]
\end{itemize}

Using the F1 score as a performance metric should reward models that are both precise and able to recall relevant rows of data effectively, ensuring a robust model. F1 score is particularly relevant for sentiment analysis with multiple labels, as it ensures a model is not just performing well on the majority class, and can accurately predict less frequent classes of data.

The three models were evaluated on approximately 1000 samples taken randomly from the unseen data, ensuring no prior knowledge from training could influence the evaluation, and providing a fair assessment of the models' performance. The unseen data was processed in the exact same way as the training data to ensure consistency.

\newpage
The configurations evaluated were the best 3 from hyper-parameter optimisation (see 4.2.4): Trial 5 (Configuration 1), Trial 1 (Configuration 2), and Trial 2 (Configuration 3). The results are as follows:

\begin{table}[ht]
    \centering
    \begin{tabular}{lcccc}
        \toprule
        Class        & Precision & Recall & F1-Score & Samples \\ 
        \midrule
        Positive     & 0.95      & 0.75   & 0.84     & 483     \\ 
        Negative     & 0.86      & 0.62   & 0.72     & 190     \\ 
        Neutral      & 0.71      & 0.99   & 0.83     & 402     \\ 
        \midrule
        Accuracy     & \multicolumn{3}{c}{0.82} & 1075 \\  
        Weighted Avg & 0.85      & 0.82   & 0.81     & 1075     \\ 
        \bottomrule
    \end{tabular}
    \caption{Classification report for Model Configuration 1.}
    \label{tab:config1}
\end{table}
\begin{table}[ht]
    \centering
    \begin{tabular}{lcccc}
        \toprule
        Class        & Precision & Recall & F1-Score & Samples \\ 
        \midrule
        Positive     & 0.96      & 0.44   & 0.61     & 483     \\ 
        Negative     & 0.87      & 0.32   & 0.46     & 190     \\ 
        Neutral      & 0.51      & 0.99   & 0.67     & 402     \\ 
        \midrule
        Accuracy     & \multicolumn{3}{c}{0.63} & 1075 \\ 
        Weighted Avg & 0.77      & 0.63   & 0.61     & 1075     \\ 
        \bottomrule
    \end{tabular}
    \caption{Classification report for Model Configuration 2.}
    \label{tab:config1}
\end{table}
\begin{table}[ht]
    \centering
    \begin{tabular}{lcccc}
        \toprule
        Class        & Precision & Recall & F1-Score & Samples \\ 
        \midrule
        Positive     & 0.94     & 0.55   & 0.69    & 483     \\ 
        Negative     & 0.81      & 0.59   & 0.69     & 190     \\ 
        Neutral      & 0.61      & 0.99   & 0.76     & 402     \\ 
        \midrule
        Accuracy     & \multicolumn{3}{c}{0.72} & 1075 \\ 
        Weighted Avg & 0.79     & 0.72   & 0.72     & 1075     \\ 
        \bottomrule
    \end{tabular}
    \caption{Classification report for Model Configuration 3.}
    \label{tab:config1}
\end{table}

The models all struggled with the negative class which is likely because the dataset contains far fewer of these samples than neutral or positive samples. They appear to mislabel them as neutral data due to the low precision of the neutral class for all configurations. The first configuration is far better than the other two with a much higher F1 score, and as such was the chosen model. The model weights were saved using the transformers library, for later use.

% -------------------------------------------------------------
% EXPECTED POINTS MODEL	
% -------------------------------------------------------------
\chapter[Expected Points Model]{Expected Points Model}
% -------------------------------------------------------------
% TEAM SELECTION ALGORITHM
% -------------------------------------------------------------
\chapter[Team Selection Algorithm]{Team Selection Algorithm}
% -------------------------------------------------------------
% RESULTS AND DISCUSSION
% -------------------------------------------------------------
\chapter[Results and Discussion]{Results and Discussion}
% -------------------------------------------------------------
% CONCLUSION AND FUTURE WORK
% -------------------------------------------------------------
\chapter[Conclusion and Future Work]{Conclusion and Future Work}
% -------------------------------------------------------------
% BIBLIOGRAPHY
% -------------------------------------------------------------
\begingroup
\raggedright
\bibliographystyle{IEEEtran}
\bibliography{Bib}
\endgroup




\end{document}